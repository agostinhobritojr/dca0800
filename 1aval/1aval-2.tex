\documentclass[12pt,a4paper]{article}
\usepackage{epsfig}
\usepackage{multicol}
%\usepackage{noitemsep}
\usepackage[utf8]{inputenc}
\usepackage[portuges]{babel}
\usepackage{fancyheadings}
\usepackage{amsmath}
\usepackage{ulem}
\usepackage{subfigure}
\usepackage{tabularx}
\usepackage{geometry}
% As margens
\geometry{left=1.0cm, top=1.0cm, bottom=1.0cm, right=1cm}
 
\renewcommand{\labelenumii}{\alph{enumii})}
\newcommand{\lacuna}{(\hspace*{5mm}) }
\thispagestyle{empty}

\begin{document}
\begin{center}\scshape
  DCA - CT - UFRN
\rule{\linewidth}{1pt}
\end{center}
 
\begin{center}\Large
DCA0800 -- Algoritmos e lógica de programação (1\raisebox{0.5ex}{\small a} avaliação - 2018.1)\\
\end{center}
\noindent Aluno:\hrulefill \vspace{2mm} Matrícula: \rule{2.5cm}{0.4pt}

\begin{enumerate}

\item (2,0 pontos) Um programador resolveu desenhar uma fileira de
  números na tela conforme um número que é fornecido no teclado, como
  mostra a saida que segue:
\begin{verbatim}
digite o numero: 6
1
22
333
4444
55555
666666
55555
4444
333
22
1
\end{verbatim}

Implemente um programa em C para realizar essa funcionalidade.

\item (1,5 pontos) 2520 é o menor número que pode ser dividido pelos números de 1 a
  10 sem deixar resto. Qual é o menor número positivo que pode ser
  divisível pelos números de 1 a 20 sem deixar resto?

\item (1,5 pontos) Prepare um programa para ler a velocidade máxima permitida em
  uma avenida e a velocidade ({\bf representada com tipos de dados
  inteiros}) com que o motorista estava dirigindo nela e calcule a
  multa que uma pessoa vai receber, sabendo que são pagos:

  \begin{enumerate}
  \item 50 reais, se o motorista estiver ultrapassar em até 10km/h a
    velocidade permitida (ex.: velocidade máxima: 50km/h; motorista a
    60km/h ou a 56km/h);
  \item 100 reais, se o motorista ultrapassar de 11 a 30 km/h a velocidade permitida.
  \item 200 reais, se estiver a partir de 31km/h acima da velocidade permitida.
  \end{enumerate}


\end{enumerate}

Embora a avaliação tenha valor igual a 4,0 pontos, a soma dos valores
das questões é igual a 5,0 pontos. Caso sua pontuação exceda os 4,0
pontos, ela será restrita a esses 4,0 pontos.  Submeta sua solução no
SIGAA. {\bf A cópia é proibida, sob pena de nulidade do exame. A
  interpretação das questões faz parte do exame.}
\begin{verbatim}
=== MODELO DE RELATORIO DE PROVA EM TXT ===
Aluno:
== questao 1 ====================
bla bla bla

== questao 2 ====================
bla bla bla
\end{verbatim}

\end{document}
