\documentclass[12pt,a4paper]{article}
\usepackage{epsfig}
\usepackage{multicol}
%\usepackage{noitemsep}
\usepackage[utf8]{inputenc}
\usepackage[portuges]{babel}
\usepackage{fancyheadings}
\usepackage{amsmath}
\usepackage{ulem}
\usepackage{subfigure}
\usepackage{tabularx}
\usepackage{geometry}
% As margens
\geometry{left=1.0cm, top=1.0cm, bottom=1.0cm, right=1cm}
 
\renewcommand{\labelenumii}{\alph{enumii})}
\newcommand{\lacuna}{(\hspace*{5mm}) }
\thispagestyle{empty}

\begin{document}
\begin{center}\scshape
  DCA - CT - UFRN
\rule{\linewidth}{1pt}
\end{center}
 
\begin{center}\Large
DCA0800 -- Algoritmos e lógica de programação (1\raisebox{0.5ex}{\small a} avaliação - 2017.1)\\
\end{center}
\noindent Aluno:\hrulefill \vspace{2mm} Matrícula: \rule{2.5cm}{0.4pt}

\begin{enumerate}

\item (1,5 pontos) Um programador resolveu desenhar uma fileira de
  números na tela conforme um número que é fornecido no teclado, como
  mostra a saida que segue:
\begin{verbatim}
digite o numero: 6
1
22
333
4444
55555
666666
55555
4444
333
22
1
\end{verbatim}

Implemente um programa em C para realizar essa funcionalidade.

\item (1,5 pontos) Elabore um programa em linguagem C para ler do teclado um número
  inteiro e decompô-lo em notação posicional, como mostra o seguinte exemplo
\begin{verbatim}
digite o numero: 321
321 = 3*10^2 + 2*10^1 + 1*10^0 = 300 + 20 + 1 = 321
digite o numero: 1302
1302 = 1*10^3 + 3*10^2 + 0*10^1 + 2*10^0 = 1000 + 300 + 2 = 1302
\end{verbatim}

\item (2,0 pontos) Elencando os seis primeiros números primos: 2, 3,
  5, 7, 11, and 13, podemos ver que o sexto primo é 13.  Elabore um
  programa em C capaz de responder corretamente a seguinte pergunta:

  - Qual é o número primo de ordem 10001?

  Insira comentários no seu código de modo a permitir o ENTENDIMENTO
  DO SEU ALGORITMO.
\end{enumerate}

Embora a avaliação tenha valor igual a 4,0 pontos, a soma dos valores
das questões é igual a 5,0 pontos. Caso sua pontuação exceda os 4,0
pontos, ela será restrita a esses 4,0 pontos.  Submeta sua solução no
SIGAA. {\bf A cópia é proibida, sob pena de nulidade do exame. A
  interpretação das questões faz parte do exame.}
\begin{verbatim}
=== MODELO DE RELATORIO DE PROVA EM TXT ===
Aluno:
== questao 1 ====================
bla bla bla

== questao 2 ====================
bla bla bla
\end{verbatim}

\end{document}
