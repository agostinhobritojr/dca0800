\documentclass{beamer}
\usepackage[utf8]{inputenc}
\usepackage[brazil]{babel}
\usepackage{listings}

\usetheme{Darmstadt}

\title{Exemplo de uso do beamer}
\author{Agostinho Brito}
\institute{
  Departamento de Engenharia da Computação e Automação\\
  Universidade Federal do Rio Grande do Norte}

\begin{document}

\frame{\titlepage}

\frame{\tableofcontents}

\section{Introdução}
\begin{frame}
  \frametitle{Meu primeiro slide}
  \begin{enumerate}
  \item Os slides do beamer funcionam com os mesmos recursos do \LaTeX
     que se usam nos documentos comuns
    \begin{itemize}
    \item José
    \item Maria
    \item João
    \end{itemize}
  \item Continua aqui
  \end{enumerate}
\end{frame}

\begin{frame}[t]
  \frametitle{Overlays}
  Um overlay é uma sequência de definições que podem surgir conforme a
  passagem dos slides.
  \begin{itemize}
  \item <2-> José
  \item <3-> Maria
  \item <1-> João
  \end{itemize}
\end{frame}

\begin{frame}
  \frametitle{Um exemplo de figura}
  \transboxin
  \begin{center}
    \includegraphics[width=0.5\linewidth]{biel.png}
    
    alor ma adf 34r dfg fdg gasdf wersdfasf asf asdf asdf qwef asdf
    qwe fasdff g  y ef sadfasdfwe  safasdf
  \end{center}
\end{frame}

\section{Desenvolvimento}
\begin{frame}
  \frametitle{Uso de blocos}
  \begin{block}{Equação de segundo grau}
    A equação de segundo grau tem a formula
    \begin{displaymath}
      f(x) = ax^2+bx+c
    \end{displaymath}
  \end{block}
  \begin{block}{Equação de primeiro grau}
    A equação de primeiro grau tem a formula
    \begin{displaymath}
      f(x) = ax+b
    \end{displaymath}
  \end{block}
\end{frame}

\begin{frame}[fragile]
  \lstinputlisting[frame=tb, caption={Ponderação de um número}]{natalidade.c}  
\end{frame}
\end{document}